\documentclass{article}
\usepackage{amsmath}
\usepackage{amssymb}
\usepackage{amsthm}
\usepackage{mathtools}
\usepackage{enumerate}
\usepackage{centernot}
\usepackage[margin=1in]{geometry}

\newcommand{\p}[1]{\left(#1\right)}
\newcommand{\f}[2]{\frac{#1}{#2}}
\newcommand{\bb}[1]{\mathbb{#1}}
\newcommand{\abs}[1]{\left\lvert#1\right\rvert}

\newcommand{\Z}{\mathbb{Z}}

\begin{document}

	\section*{Arithmetic progressions in a subset of $\Z/n\Z$ and its complement}

	Let $n > 3$ be an odd. Consider a subset $S \subseteq \Z/n\Z$. Define $A(S)$ to be the number of length 3 arithmetic progressions in $S$. That is,
	\[ A(S) = \abs{\{a, b \in \Z/n\Z \mid b \neq 0;\ a, a + b, a + 2b \in S \}}. \]
	Note: this method of counting considers $(x, y, z)$ and $(z, y, x)$ to be different arithmetic progression, and does not consider $(x, x, x)$ to be an arithmetic progression. However, the result still holds under every permutation of these settings.

	Let $k = \abs{S}$, and denote the complement of $S$ in $\Z/n\Z$ as $S^c$, so $\abs{S^c} = n - k$. The total number of arithmetic progressions $A(\Z/n\Z)$ is $n(n-1)$. These arithmetic progressions can be split into four groups based on the locations of their elements: all in $S$, all in $S^c$, exactly one in $S$, and exactly one in $S^c$.

	There are $A(S)$ progressions all in $S$ and $A(S^c)$ all in $S^c$. The number of progressions with exactly one element in $S$ is $3k(k-1) - 3A(S)$ since each ordered pair of elements of $S$ is contained in $3$ total progressions, but we need to subtract the contribution of the progressions entirely contained in $S$, each of which contains $3$ ordered pairs. Similarly, the number of progressions with exactly one element in $S^c$ is $3(n-k)(n-k-1) - 3A(S^c)$. Adding everything together, we get
	\[ n(n-1) = A(S) + (3k(k-1) - 3A(S)) + (3(n-k)(n-k-1) - 3A(S^c)) + A(S^c). \]
	Simplifying,
	\[ A(S) + A(S^c) = \f{1}{2}\p{3k(k-1) + 3(n-k)(n-k-1) - n(n-1)}. \]
	Surprisingly, this sum depends only on $n$ and $k$.

\end{document}
